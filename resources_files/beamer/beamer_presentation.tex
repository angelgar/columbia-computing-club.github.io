\documentclass{beamer}
\usepackage[utf8]{inputenc}

\usetheme{Madrid}
\usecolortheme{default}

%------------------------------------------------------------
%This block of code defines the information to appear in the
%Title page
\title[Beamer Intro] %optional
{Presentations in Beamer}

\subtitle{The basics}

\author[Computing Club] % (optional)
{Safiya Sirota\inst{1} \and Charly Fowler\inst{2}}

\institute[CUIMC] % (optional)
{
  \inst{1}
  Third year biostats PhD student\\
  Columbia University

  \inst{2}
  Fifth year biostats PhD student\\
  Columbia University
  
}

\date[Spring 2024] % (optional)
{Computing Club, April 2024}

\logo{\includegraphics[height=1cm]{logo.jpeg}}

%End of title page configuration block
%------------------------------------------------------------



%------------------------------------------------------------
%The next block of commands puts the table of contents at the 
%beginning of each section and highlights the current section:

\AtBeginSection[]
{
  \begin{frame}
    \frametitle{Outline}
    \tableofcontents[currentsection]
  \end{frame}
}
%------------------------------------------------------------


\begin{document}

\begin{frame}
\frametitle{Sign in!}

Please scan this QR code and sign in :)

\begin{figure}[H]
    \centering
    \includegraphics[scale = 0.5]{frame.png}
\end{figure}

\end{frame}

%The next statement creates the title page.
\frame{\titlepage}


%---------------------------------------------------------
%This block of code is for the table of contents after
%the title page
\begin{frame}
\frametitle{Outline}
\tableofcontents
\end{frame}
%---------------------------------------------------------


\section{What is Beamer?}

%---------------------------------------------------------
%Changing visivility of the text
\begin{frame}
\frametitle{What is \LaTeX?}

\begin{itemize}
    \item<1-> \LaTeX \hspace{0.05cm} is software for typesetting documents
    \item<2-> It is widely used in the scientific community
    \item<3> Once you have the software, you can write \LaTeX \hspace{0.05cm} scripts in Overleaf, RMarkdown, text files, and other environments
\end{itemize}
\end{frame}

%---------------------------------------------------------


%---------------------------------------------------------

\begin{frame}
\frametitle{What is Beamer?}

\begin{itemize}
    \item<1-> Beamer is a document "class" in \LaTeX
    \item<2-> I'm using it right now to make these slides
    \item<3-> Typically, the "article" class is used for creating papers and "beamer" is used for presentations
\end{itemize}
\end{frame}

%---------------------------------------------------------


%---------------------------------------------------------

\begin{frame}
\frametitle{Why use Beamer?}

\begin{itemize}
    \item<1-> Beamer, like the "article" class, typesets
    \item<2-> This means that once you understand how to use \LaTeX, you don't have to worry about formatting your slides
    \item<3-> Presentation creation goes much faster if you have a lot of mathematical symbols in your content
    \item<4> Many people in our field and adjacent ones use Beamer to create their presentations
\end{itemize}
\end{frame}

%---------------------------------------------------------


%---------------------------------------------------------

\begin{frame}
\frametitle{Why not use Beamer?}

\begin{itemize}
    \item<1-> When you want to create a \alert{flashy} presentation
    \item<2-> When you want a lot of control over the formatting and don't want it to be uniform across slides
    \item<3-> When you don't have many equations
    \item<4> When you're in a rush and you haven't yet mastered Beamer
\end{itemize}
\end{frame}

%---------------------------------------------------------

\section{Beamer basics and tips}

%---------------------------------------------------------
\begin{frame}
\frametitle{Tips for getting started}

\begin{itemize}
    \item<1-> It's always faster to start with a pre-made template; there's no need to begin with a blank document
    \item<2-> We have plenty of resources on \LaTeX \hspace{0.05cm} Beamer from previous years' \alert{\href{https://cu-biostats-computing-club.org/resources.html}{computing club meetings }}
    \item<3-> Creating your presentation in \alert{\href{http://www.overleaf.com}{Overleaf}} may be helpful at first so that you can compile quickly to check that your document looks how you want
    \item<4-> Leave yourself time to make your presentation if you are new to \LaTeX; there is a learning curve
    \item<5> For this presentation, I started with this \alert{\href{https://www.overleaf.com/learn/latex/Beamer}{template}}
\end{itemize}

\end{frame}

%---------------------------------------------------------

\section{Beamer features}

%---------------------------------------------------------

\begin{frame}
\frametitle{What can we do in Beamer?}

\begin{itemize} 
    \item<1-> Easily create a title page \pause
    \item<2-> Easily create a outline \pause
    \item<3->Set a theme \pause
    \item<4-> Decide when content shows on the slide \pause
    \item<4-> Insert equations \pause
    \item<5-> Embed images \pause
    \item<6-> Format content into multiple columns \pause
\end{itemize} 
 
\hspace{1cm}
 
I'll show what the last 3 things look like in the presentation.  \pause The first few functions I've already used throughout, but I'll show you what "coding" it looks like in a \textbf{.tex} file.

\end{frame}

%---------------------------------------------------------



%---------------------------------------------------------

\begin{frame}
\frametitle{Inserting equations}

\begin{itemize}
    \item<1-> Writing equations in Beamer is one of the easiest things you can do once you know \LaTeX! \pause
    \item<2-> There are a few different ways to make equations \pause
\end{itemize}

\hspace{0.5cm}

The first way is with in-line text, like so, \pause $f(x| \mu, \sigma) = \frac{1}{\sqrt{2\pi}\sigma}\exp\{-\frac{1}{2}(\frac{x-\mu}{\sigma})^2\}$. \pause

\hspace{0.5cm}

Another way is by using the equation environment: \pause

\begin{equation}
f(x| \mu, \sigma) = \frac{1}{\sqrt{2\pi}\sigma}\exp\{-\frac{1}{2}(\frac{x-\mu}{\sigma})^2\}
\end{equation}

\end{frame}

%---------------------------------------------------------



%---------------------------------------------------------

\begin{frame}
\frametitle{Inserting equations}

You can also make the equation show up without a number,

\begin{equation*}
f(x| \mu, \sigma) = \frac{1}{\sqrt{2\pi}\sigma}\exp\{-\frac{1}{2}(\frac{x-\mu}{\sigma})^2\}
\end{equation*} \pause

\hspace{0.5cm}

or in multiple lines:

\begin{align*}
f(x| \mu, \sigma) &= \frac{1}{\sqrt{2\pi}\sigma}\exp\{-\frac{1}{2}(\frac{x-\mu}{\sigma})^2\} \\
&= \frac{1}{\sqrt{2\pi}}\exp \left\{-\frac{(x-\mu)^2}{2\sigma^2}\right\}
\end{align*}

\end{frame}

%---------------------------------------------------------



%---------------------------------------------------------

\begin{frame}
\frametitle{Inserting equations}

\begin{block}{Normal block}
Finally, to make equations stand out, you can use these special blocks. \pause
\end{block}

\begin{alertblock}{Alert block}
The pdf for a R.V. X that follows a $N(\mu, \sigma^2)$ distribution is as follows. \pause
\end{alertblock}

\begin{examples}
\begin{equation*}
f(x| \mu, \sigma) = \frac{1}{\sqrt{2\pi}}\exp \left\{-\frac{(x-\mu)^2}{2\sigma^2}\right\}
\end{equation*}
\end{examples}

\end{frame}

%---------------------------------------------------------



%---------------------------------------------------------

\begin{frame}
\frametitle{Embedding images}

Adding images to your Beamer presentation is also relatively simple. \pause It does the formatting for you, so you just have to make sure your images are a good size for the page. \pause


\begin{figure}[b]
    \centering
    \includegraphics[scale = 0.03]{tree.jpg}
    \caption{Tree in my backyard}
\end{figure}


\end{frame}

%---------------------------------------------------------



%---------------------------------------------------------

\begin{frame}
\frametitle{Slides with multiple columns}

\begin{columns}

\column{0.5\textwidth}
\begin{itemize}
\item<1-> One of the popular slide layouts in Powerpoint is a slide with multiple columns 
\item<2-> Usually you may want 2 or 3 
\end{itemize}

\column{0.5\textwidth}
\begin{itemize}
    \item<3-> Luckily, you can do this in \LaTeX \hspace{.05cm} as well.
    \item<4-> This is a slide with 2 columns
\end{itemize}
\end{columns}
\end{frame}


%---------------------------------------------------------



%---------------------------------------------------------

\begin{frame}
\frametitle{Slides with multiple columns}

\begin{columns}

\column{0.3\textwidth}
Now let's try a slide with 3 columns. The first column here just has a text block. \pause

\column{0.3\textwidth}
\begin{itemize}
    \item<2-> This column here has a couple of bullet points \pause
    \item<3-> The next column will have a picture \pause
\end{itemize}

\column{0.3\textwidth}
\begin{figure}[h]
    \centering
    \includegraphics[scale = 0.025]{tree.jpg}
\end{figure}

\end{columns}
\end{frame}

%---------------------------------------------------------

\section{Writing a \textbf{.tex} script in Overleaf}

%---------------------------------------------------------

\begin{frame}
\frametitle{What does the \textbf{.tex} file look like?}

\hspace{1cm}

Now, I'll show you what everything looks like when written in a \textbf{.tex} file in Overleaf!

\end{frame}

%---------------------------------------------------------

\section{Making a Beamer presentation in Rmarkdown}

%---------------------------------------------------------

\begin{frame}
\frametitle{Using RMarkdown}

\hspace{1cm}

Now, I'll show you very briefly what setting up a Beamer presentation in RMarkdown looks like.

\end{frame}


\end{document}